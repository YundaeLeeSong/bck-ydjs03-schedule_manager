\documentclass[12pt]{article}
\usepackage{amsmath}
\usepackage{amssymb}
\usepackage{geometry}
\geometry{margin=1in}

\begin{document}

\section*{Newton’s Second Law (Dynamics)}

Newton’s second law relates the net external force acting on an object to its mass and acceleration. Formally:
\begin{equation}
\sum \mathbf{F} = m\,\mathbf{a},
\end{equation}
where:
\begin{itemize}
    \item $\sum \mathbf{F}$ (often written as $\mathbf{F}_{\mathrm{net}}$) is the vector sum of all external forces,
    \item $m$ is the mass of the object (a scalar),
    \item $\mathbf{a}$ is the acceleration vector of the object.
\end{itemize}
Equivalently, one can write
\begin{equation}
\mathbf{a} \;=\; \frac{\sum \mathbf{F}}{m}.
\end{equation}
Thus, the acceleration of an object is directly proportional to the net force and inversely proportional to its mass.

\subsection*{1. Constant Force (Uniform Acceleration)}
When a single constant force $\mathbf{F}$ acts on an object of mass $m$, the acceleration is constant:
\begin{equation}
\mathbf{F} = m\,\mathbf{a}
\quad\Longrightarrow\quad
\mathbf{a} = \frac{\mathbf{F}}{m} = \text{constant}.
\end{equation}
Under uniform acceleration, the velocity and position evolve as:
\begin{align}
\mathbf{v}(t) &= \mathbf{v}_0 + \mathbf{a}\,t, \\
\mathbf{r}(t) &= \mathbf{r}_0 + \mathbf{v}_0\,t + \tfrac{1}{2}\,\mathbf{a}\,t^2,
\end{align}
where $\mathbf{v}_0$ and $\mathbf{r}_0$ are the initial velocity and position, respectively.

\subsubsection*{Example: Object in a Uniform Gravitational Field}
Near Earth’s surface, an object of mass $m$ experiences a constant gravitational force:
\begin{equation}
\mathbf{F}_{g} = -\,m\,g\,\hat{y}, 
\end{equation}
where $g \approx 9.81~\mathrm{m/s^2}$ and $\hat{y}$ is the unit vector in the upward direction. By Newton’s second law:
\begin{align}
\sum \mathbf{F} &= \mathbf{F}_{g} = -\,m\,g\,\hat{y}, \\
m\,\mathbf{a} &= -\,m\,g\,\hat{y}, \\
\mathbf{a} &= -\,g\,\hat{y} = \text{constant}.
\end{align}
Hence, the object accelerates downward at $g$, and its kinematic equations (in one dimension) become:
\begin{align}
v_y(t) &= v_{y0} - g\,t, \\
y(t)  &= y_0 + v_{y0}\,t - \tfrac{1}{2}\,g\,t^2.
\end{align}

\subsection*{2. Multiple Forces (Net Force Calculation)}
If several forces act simultaneously, sum their vector contributions to find $\mathbf{F}_{\mathrm{net}}$:
\begin{equation}
\sum \mathbf{F} = \mathbf{F}_1 + \mathbf{F}_2 + \cdots + \mathbf{F}_n = m\,\mathbf{a}.
\end{equation}
In such scenarios, one often decomposes forces into components (e.g., along $\hat{x}$ and $\hat{y}$) to solve for acceleration in each direction.

\subsubsection*{Example: Block on an Inclined Plane with Friction}
Consider a block of mass $m$ on an incline at angle $\theta$ above horizontal. Forces acting on the block include:
\begin{itemize}
    \item Gravitational force:
    \[
    \mathbf{F}_{g} = -\,m\,g\,\hat{j}.
    \]
    \item Normal force (perpendicular to plane):
    \[
    \mathbf{F}_N = +\,m\,g\,\cos\theta \,\hat{n},
    \]
    where $\hat{n}$ is the unit vector normal to the surface.
    \item Kinetic friction (if sliding), directed up the plane:
    \[
    \mathbf{F}_{f} = -\,\mu_{k}\,F_N\,\hat{t} 
    = -\,\mu_{k}\,(m\,g\,\cos\theta)\,\hat{t},
    \]
    where $\mu_{k}$ is the coefficient of kinetic friction and $\hat{t}$ is the unit vector tangent (down the plane).
\end{itemize}
Decomposing gravitational force along and perpendicular to the plane:
\begin{align}
\mathbf{F}_{g,\parallel} &= -\,m\,g\,\sin\theta \,\hat{t}, \\
\mathbf{F}_{g,\perp}    &= -\,m\,g\,\cos\theta \,\hat{n}.
\end{align}
Since $\mathbf{F}_N$ balances $\mathbf{F}_{g,\perp}$, the net perpendicular force is zero:
\[
\sum F_{\perp} = F_{N} + F_{g,\perp} = m\,g\,\cos\theta \;-\; m\,g\,\cos\theta = 0.
\]
Hence, no acceleration normal to the plane ($a_{\perp} = 0$). Along the plane, the net force is:
\begin{equation}
\sum F_{\parallel} 
= F_{g,\parallel} + F_{f} 
= \bigl(-\,m\,g\,\sin\theta \bigr) 
  + \bigl(-\,\mu_{k}\,m\,g\,\cos\theta \bigr) 
= -\,m\,g\,\bigl(\sin\theta + \mu_{k}\cos\theta \bigr).
\end{equation}
Applying Newton’s second law along $\hat{t}$:
\begin{align}
m\,a_{\parallel} 
&= -\,m\,g\,\bigl(\sin\theta + \mu_{k}\cos\theta \bigr), \\
a_{\parallel} 
&= -\,g\,\bigl(\sin\theta + \mu_{k}\cos\theta \bigr).
\end{align}
The negative sign indicates the block accelerates down the plane if released from rest.

\subsection*{3. Variable Force (Non‐Uniform Acceleration)}
When a force varies with position, velocity, or time, acceleration is also non‐constant. In the most general form:
\begin{equation}
\sum \mathbf{F}(t,\,\mathbf{r},\,\mathbf{v}) = m\,\mathbf{a}(t),
\end{equation}
where $\mathbf{F}$ may be a function of time $t$, position $\mathbf{r}$, and velocity $\mathbf{v}$. One then solves a differential equation:
\begin{equation}
m\,\frac{d^2 \mathbf{r}}{dt^2} = \mathbf{F}\bigl(t,\mathbf{r}(t),\mathbf{v}(t)\bigr).
\end{equation}

\subsubsection*{Example: Drag Force on a Falling Object}
A small object (e.g., a raindrop) falling through air experiences a drag force approximately proportional to the square of its speed:
\begin{equation}
\mathbf{F}_{d} = -\,c_{d}\,v^2\,\hat{v},
\end{equation}
where $c_{d}$ is a drag coefficient, $v = \|\mathbf{v}\|$, and $\hat{v} = \mathbf{v}/v$. The net force in the vertical ($\hat{y}$) direction:
\begin{equation}
\sum F_{y} = -\,m\,g \;-\; c_{d}\,v^2,
\end{equation}
with downward taken as negative. By Newton’s second law:
\begin{equation}
m\,\frac{dv}{dt} = -\,m\,g \;-\; c_{d}\,v^2.
\end{equation}
This yields a non‐linear ordinary differential equation that can be solved (analytically or numerically) to find $v(t)$, eventually approaching terminal velocity when $m\,dv/dt = 0$.

\subsection*{Summary}
Newton’s second law quantitatively links force, mass, and acceleration:
\[
\boxed{\sum \mathbf{F} \;=\; m\,\mathbf{a}.}
\]
\begin{itemize}
    \item For a constant net force, acceleration is uniform: $\mathbf{a} = \mathbf{F}/m$.
    \item When multiple forces act, sum their vector contributions to find $\mathbf{F}_{\mathrm{net}}$, then compute $\mathbf{a} = \mathbf{F}_{\mathrm{net}}/m$.
    \item If forces vary with time, position, or velocity, one must solve $m\,d^2\mathbf{r}/dt^2 = \mathbf{F}(t,\,\mathbf{r},\,\mathbf{v})$ to determine the motion.
\end{itemize}

\end{document}
