\documentclass[12pt]{article}
\usepackage{amsmath}
\usepackage{amssymb}
\usepackage{geometry}
\geometry{margin=1in}

\begin{document}

\section*{Newton’s First Law (Inertia)}

In the absence of any net external force, an object maintains its current state of motion—either remaining at rest or moving with constant velocity. Formally, if the sum of all external forces acting on an object is zero, then:
\begin{equation}
\sum \mathbf{F} = \mathbf{0} \quad \Longrightarrow \quad \mathbf{a} = \mathbf{0},
\end{equation}
where \(\sum \mathbf{F}\) (often written \(\mathbf{F}_{\mathrm{net}}\)) is the vector sum of all forces, and \(\mathbf{a}\) is the acceleration. According to Newton’s second law,
\begin{equation}
\mathbf{F}_{\mathrm{net}} = m\,\mathbf{a},
\end{equation}
so setting \(\sum \mathbf{F} = \mathbf{0}\) implies
\begin{equation}
m\,\mathbf{a} = \mathbf{0} \quad \Longrightarrow \quad \mathbf{a} = \mathbf{0}.
\end{equation}
Since \(\mathbf{a} = \frac{d\mathbf{v}}{dt}\), it follows that
\begin{equation}
\frac{d\mathbf{v}}{dt} = \mathbf{0} \quad \Longrightarrow \quad \mathbf{v}(t) = \text{constant}.
\end{equation}

\subsection*{Example 1: Frictionless Puck}
Consider an air hockey puck sliding on an ideal frictionless surface. Once the puck is given an initial push and set into motion, there are no horizontal (tangential) forces opposing its motion (neglecting air resistance). Therefore:
\begin{align}
\mathbf{F}_{\mathrm{net}} &= \mathbf{0} \\
m\,\mathbf{a} &= \mathbf{0} \\
\mathbf{a} &= \mathbf{0} \\
\mathbf{v}(t) &= \text{constant}.
\end{align}
Thus, the puck continues moving in a straight line at constant speed indefinitely.

\subsection*{Example 2: Book on a Horizontal Table}
A book of mass \(m\) rests on a perfectly horizontal table. Two forces act on the book:
\begin{itemize}
    \item Gravitational force (weight), directed downward:
    \[
    \mathbf{F}_{g} = -\,m\,g\,\hat{y},
    \]
    where \(g \approx 9.81\ \text{m/s}^2\) and \(\hat{y}\) is chosen as the unit vector pointing upward.
    \item Normal force from the table, directed upward:
    \[
    \mathbf{F}_{N} = +\,m\,g\,\hat{y}.
    \]
\end{itemize}
Since these two forces are equal in magnitude and opposite in direction:
\begin{equation}
\sum \mathbf{F} = \mathbf{F}_{N} + \mathbf{F}_{g} 
         = m\,g\,\hat{y} \;-\; m\,g\,\hat{y} 
         = \mathbf{0},
\end{equation}
which implies
\begin{equation}
\mathbf{a} = \mathbf{0} 
\quad \Longrightarrow \quad
\mathbf{v}(t) = \mathbf{0}
\quad \Longrightarrow \quad
\text{Book remains at rest.}
\end{equation}

\subsection*{Summary}
Inertia is the property of matter that causes it to resist changes in motion. Newton’s first law mathematically states that if no net external force acts on an object (\(\sum \mathbf{F} = \mathbf{0}\)), then that object’s acceleration must be zero (\(\mathbf{a} = \mathbf{0}\)), and its velocity remains constant. Whether an object is initially at rest or in motion, it will preserve that state until acted upon by a nonzero resultant force.

\end{document}
